Mobile devices have become a huge part of our lives in the last decade, when they've continuously evolved and are now capable of accessing the internet, banking, taking pictures with high quality, and many other functionalities. The Android platform is the most used mobile platform in the world, with 71.63\% market share and 3.6 billion users worldwide \cite{turner_how_nodate}. Android applications (or \textit{apps}) are distributed in many marketplaces, with Google Play Store being the largest and most used having over 3.5 million apps available \cite{turner_how_nodate-1}. The huge popularity combined with the access to many sensitive capabilities have turned mobile devices on attractive targets for malicious actors: Truong et al. have reported than over 0.25\% of Android devices were infected with malware \cite{truong_company_2014}.

% falar sobre repackaging class, citar MAS pra resolver esses caras
Most Android malware work using a type of attack called \textit{repackaging}, which consists of altering an existing application and introducing malicious behavior on it. The repackaged apps are distributed in many marketplaces, and trick users into thinking that it's the original version of the app. Once these harmful applications are installed into a device, they're capable of capturing user input, recording data from the microphone or camera and sending it to third party servers.

For that reason, previous works studied a method called \textit{mining sandboxes}, first proposed by Jamrozik et al. \cite{jamrozik_mining_2016} that consists on capturing an app behavior using automated test generation tools. This technique was shown by Bao et al \cite{bao_mining_2018} to be an effective tool to detect malicious activity on Android apps, while comparing the effectiveness of multiple test case generation tools. Bao et al's study has then been complemented by Costa et al. \cite{costa_exploring_2022}, that built a tool called DroidXP to combine both static analysis and the dynamic analysis of the mining sandboxes approach and achieve a higher effectiveness on malware detection. Le et al.  \cite{le_towards_2018} have also extended Jamrozik's work by creating a more robust sandbox that takes into account not only what different behaviors are introduced by the repackaged version of the apps, but also what's different on existing behaviors (specifically, different arguments passed to sensitive methods).

In this study, we'll explore this technique a little bit further. Previous studies have used a small dataset of pairs of repackaged apps (102 pairs for \cite{bao_mining_2018} and \cite{costa_exploring_2022}, and 25 pairs for \cite{le_towards_2018}), which may compromise the external validity. For that reason, this study will evaluate the performance of combining the static analysis and dynamic analysis of the mining sandboxes approach, as studied by Costa \cite{costa_exploring_2022} with Le's proposal of comparing the arguments input into sensitive APIs \cite{le_towards_2018} for detecting Android repackaged malware on a much larger dataset of 1,707 pairs. This will be done by adapting Costa's DroidXP \cite{costa_droidxp_2020} framework to be able to instrument Android apps and capture the arguments passed to a given set of methods.

% falar da avaliacao, falar como performou, e como a familia influencia
The contribution of this work is the evolution of the DroidXP framework and the evaluation of this new approach on the accuracy of the malware detection capabilities of the tool. The results indicate that there's a slight improvement of 3.51\% on the overall accuracy if compared to the previous version, and the detection capabilities of this new approach are highly dependant on the malware family, given that some families have a huge jump on detection while some others express no relevant improvement.