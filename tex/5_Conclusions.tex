This study shows that there's a lot of room for improvement regarding the malware detection capabilities of the mining sandboxes technique. Our main goal of improving its accuracy by adding a new criteria for malware detection was successfully achieved even though the overall improvement is modest and highly dependant on the malware family that's being analyzed.

The study was performed on a much larger dataset of 1,707 app pairs, if compared to previous studies that were often limited on 100 or less samples. It is noticeable that the malware detection performance of the mining sandboxes is also compromised when analyzing a larger dataset, even with the improvements proposed in this paper.

Future work might explore ways to improve the argument detection, since the method calls might be executed in many ways on the programming languages used by the Android platform, such as reflection or callbacks that weren't handled on the version of the tool implemented for this study, and might also  explore a different set of sensitive API calls for the analysis.

The Mining Android Sandboxes approach has been proven to be a useful tool for malware detection on the Android ecosystem with its scalability and automation capabilities, but a lot of research is still necessary in order to turn it into a reliable tool for protecting app marketplaces, devices, and final users from malicious actors.