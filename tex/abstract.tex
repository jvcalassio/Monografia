The Android platform, with its extensive user base and popularity has become a prime target for malware attacks. For that reason, researchers have been interested on malware detection methods, including the \textit{mining sandboxes} approach. This approach focuses on  \textit{repackaged} apps, a type of attack that consists on modifying an existing app and introducing malicious behavior on it, and is highly prevalent on the Android platform. The mining sandboxes technique takes advantage of test case generation tools to monitor an app's runtime behavior and detect potential malicious intentions through behavioral differences between different versions of apps. While the studies have shown promising conclusions, with over 70\% detection accuracy, there's still a lot of room for improvement.

This study investigates how the performance of the mining sandboxes can be improved by combining some previously proposed techniques (such as static analysis) with an approach that extends the behavioral differences detection by taking into consideration the arguments passed to sensitive methods, and if the type of malware has any influence on the detection effectiveness. This is done by evolving DroidXP, an existing research framework for mining sandboxes, and evaluating its performance on a comprehensive dataset of 1,707 pairs of apps. The results show that there's an improvement of 14\% on the malware detection accuracy, and that there's a high influence of the type of malware on the detection outcome.